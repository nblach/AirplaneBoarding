\documentclass[11pt]{article}
\usepackage{geometry}                
\geometry{letterpaper}                   





\usepackage[english]{babel}
\usepackage[utf8x]{inputenc}
\usepackage{amsmath}
\usepackage{tikz}
\usetikzlibrary{positioning}

\usetikzlibrary{arrows,automata}


%\usepackage[latin1]{inputenc}
\usepackage{graphicx}
\usepackage{amssymb}
\usepackage{epstopdf}
\usepackage{natbib}
\usepackage{amssymb, amsmath}
\DeclareGraphicsRule{.tif}{png}{.png}{`convert #1 `dirname #1`/`basename #1 .tif`.png}

\title{Airplane Boarding}
\author{Anton Schäfer, Nils Blach}
%\date{date} 

\begin{document}



\thispagestyle{empty}

\begin{center}
\includegraphics[width=5cm]{ETHlogo.pdf}

\bigskip


\bigskip


\bigskip


\LARGE{ 	Lecture with Computer Exercises:\\ }
\LARGE{ Modelling and Simulating Social Systems\\}

\bigskip

\bigskip

\small{Project Report}\\

\bigskip

\bigskip

\bigskip

\bigskip


\begin{tabular}{|c|}
\hline
\\
\textbf{\LARGE{Airplane Boarding}}\\
\textbf{\LARGE{...}}\\
\\
\hline
\end{tabular}
\bigskip

\bigskip

\bigskip

\LARGE{Anton Sch{\"a}fer, Nils Blach}



\bigskip

\bigskip

\bigskip

\bigskip

\bigskip

\bigskip

\bigskip

\bigskip

Zurich\\
Dec 2018\\

\end{center}



\newpage

%%%%%%%%%%%%%%%%%%%%%%%%%%%%%%%%%%%%%%%%%%%%%%%%%

\newpage
\section*{Agreement for free-download}
\bigskip


\bigskip


\large We hereby agree to make our source code for this project freely available for download from the web pages of COSS. Furthermore, we assure that all source code is written by ourselves and is not violating any copyright restrictions.

\begin{center}

\bigskip


\bigskip


\begin{tabular}{@{}p{3.3cm}@{}p{6cm}@{}@{}p{6cm}@{}}
\begin{minipage}{3cm}

\end{minipage}
&
\begin{minipage}{6cm}
	\vspace{2mm} \large Anton Sch{\"a}fer

 \vspace{\baselineskip}

\end{minipage}
&
\begin{minipage}{6cm}

\large Nils Blach

\end{minipage}
\end{tabular}


\end{center}
\newpage

%%%%%%%%%%%%%%%%%%%%%%%%%%%%%%%%%%%%%%%



% IMPORTANT
% you MUST include the ETH declaration of originality here; it is available for download on the course website or at http://www.ethz.ch/faculty/exams/plagiarism/index_EN; it can be printed as pdf and should be filled out in handwriting


%%%%%%%%%% Table of content %%%%%%%%%%%%%%%%%

\tableofcontents

\newpage

%%%%%%%%%%%%%%%%%%%%%%%%%%%%%%%%%%%%%%%



\section{Abstract}

\section{Individual contributions}
\paragraph{Model:} Both
\paragraph{Data Collection/Calculation:} Anton Sch{\"a}fer
\paragraph{Data Display/Graphs:} Nils Blach
\paragraph{Paper:} Both

\section{Introduction and Motivations}

\section{Description of the Model}

\subsection{Overview}

Our model realistically simulates the process of airplane boarding. It aims to provide reasonable information about how long it takes until all people are seated and how long it takes until a specific person is seated, depending on different boarding methods, luggage loads, and passenger count.  In order to achieve this, it focuses on actions taking place in the aisle, such as passengers moving, passengers storing items, and passengers moving into their seat.


\subsection{The Boarding Process}
\paragraph{At the gate}
Before boarding, passengers usually wait at the gate. When boarding is announced, they start to line up either in random order, or by group, depending on the airline or flight. Commonly used grouping schemes are either by blocks of seats, or by letter (A,B,...,F). The boarding agent announces which group is next to board the plane, so the passengers will enter the plane in the order of their assigned groups. Still, the order in which passengers from the same block enter the plane is random. 

\paragraph{Entering the plane}
Passengers get to the plane via bus or through a jet bridge and enter the plane through either the front door or back door (or some middle door, if the airplane is big), or through both doors. Once inside the plane, passengers usually first try to find their seat.  Before sitting down, however, they may need to store their hand luggage. 
\paragraph{Hand luggage} The three main types of hand luggage passengers carry are purses, backpacks and suitcases. Purses and some backpacks are usually stored under the seat in front, while suitcases and other backpacks need to be put in the overead compartments. When searching for an overhead compartment to store their items, passengers generally first check, if there space in a compartment right above their seat. When they see that it is full already, they usually store their luggage in the next free compartment they see. However, passengers usually avoid walking back into the direction they came from, as this would mean going against the flow of passengers in the narrow aisle.

\paragraph{Entering seat} After storing all their big hand luggage, passengers walk to their row in order to sit down. If noone sits inbetween them and their seat, they can sit down. Otherwise, the sitting passengers first have to get out of the row, in order to let in the arriving passenger.
\\\\
Once all passengers are seated, boarding is completed.

\subsection{Our model}

We want to gain insight on the efficiency of various boarding methods, and the impact of luggage load and passenger count on the time it takes individuals and all passengers to board a plane. Thus, what happens in the plane is crucial, while what is taking place in the airport, jet bridge, and bus barely have significant effects on the boarding time, other than determining the order in which people arrive at the plane door. We therefore decided to only model the events taking place inside the airplane. In order to still take into account the impact of different boarding strategies chosen by the gate agent, we assume the passengers all enter the plane in a certain order given by their seats. We will call such an order (of seats/passengers by seat) a boarding sequence.
Although we have to keep track of which passengers are already sitting, it is not relvevant what the sitting passengers are doing. Hence, for our model, we focus only on what is happening in the aisle and discard most other irrelevant parts of the plane.

We only consider airplanes with one door at the front, as all other airplanes with $n \geq 2$ doors can be divided in $n$ sections with one door. Then, boarding can simultaneously be modeled in each section individually, yielding the same result, if one assumes, that in reality, passengers entering the plane in one section will never go to a different section.
When choosing reasonable sections and assuming passengers behave rational, this can indeed be guaranteed. Similarly, we only consider airplanes with one aisle, as all other planes can again be divided into sections with one aisle each. Additionally, the plane used for modelling is configureable. Thus, our model applies to almost all airplanes (when dividing them in sections if necessary).

We use an actor based model, where the actors are passengers. For simplicity, we divide the continuous time interval in discrete timesteps. The actors move around in the aisle, which is a one dimensional line divided in discrete space units. Two actors can never be at the same position in the aisle, unless they are moving in opposite directions and have to pass each other. Then, they can switch positions. We model the boarding process of an actor as follows:

\paragraph{Entering the plane}
An actor enters the plane as soon as all actors that he is behind of in the boarding sequence  have entered the plane and if there is enough free space in the aisle for a new actor to enter. The actor then appears at the very start of the aisle.




\paragraph{Hand luggage}
For each actor, we only consider the pieces of luggage they need to store in an overhead compartment. We discard their other luggage, as it will be stored underneath the front seat, and thus has a negligible impact on boarding time. Also, instead of taking into account every single overhead compartment, for any pair of opposite compartments, our model uses one single compartment that holds twice as many items, and covers the same space of the aisle. This simplification is realistic, as from each position in the aisle, passengers can reach the compartments on both sides. Because of this abstraction, in the remainder of this paper, when referring to a "compartment" while speaking about our model, we mean one single big compartment corresponding to two opposite actual compartments.

	As soon as an actors can see their seat, they determine if they can store all their luggage in the comparment above their seat. If they can, they will store their luggage there, otherwise, they will store as much of their luggage as possible in the next free compartments they see. However, at first, they only move forward in order not to go against the flow. Only when they arrive at the very end of the plane and can still not store their luggage, they search for a free compartment towards the front of the plane.


\paragraph{Entering seat}
After storing all their luggage, the actors go to their seat in order to sit down. If other sitting passengers are obstructing the way to their seat, these passengers need to move out of their seat first in order to let in the arriving actor. The arriving actor blocks the aisle during this whole time.
\\\\
Boarding is completed as soon as every actor entered their seat.


\subsection{Our Model vs. Van Landeghem's and Beuselinck's Model} 







\section{Implementation}
\begin{figure}[h!]
	\center
\begin{tabular}{|ll|l|}
	\hline
	State & &Description\\
	\hline
0 &     & not yet in plane                        \\
\hline
1 & 1/0 & looking for storage room by seat        \\
  & 1/1 & looking for storage room behind seat    \\
  & 1/2 & looking for storage room further front  \\
  \hline
2 & 2/0 & storing luggage (coming from state 1/0) \\
  & 2/1 & storing luggage (coming from state 1/1) \\
  & 2/2 & storing luggage (coming from state 1/2) \\
  \hline
3 &     & going to seat                           \\
\hline
4 &     & sitting down                            \\
\hline
5 &     & sitting in seat       \\                 
\hline

\end{tabular}
\caption{States an actor can be in}
\label{tab:states}
\end{figure}

\begin{figure}[h!]

	\includegraphics[width=\linewidth]{images/fsm.png}
	\label{fig:fsm}
	\caption{Finite state machine describing the states an actor can be in.}
%\begin{tikzpicture}[->,>=stealth',shorten >=1pt,auto,node distance=3.5cm,
%        scale = 1,transform shape]

%\node[state] (0) [] {$0$};
% \node[state] (1/0) [above of=0] {$1/0$};
% \node[state] (1/1) [above of=1/0] {$1/1$};
% \node[state] (1/2) [above of=1/1] {$1/2$};
% \node[state] (2/0) [right of=1/0] {$2/0$};
% \node[state] (2/1) [right of=1/1] {$2/1$};
% \node[state] (2/2) [right of=1/2] {$2/2$};
% \node[state] (3) [right of=2/1] {$3$};
% \node[state] (4) [right of=3] {$4$};
% \node[state] (5) [right of=4] {$5$};
%
% \path (0) edge   [bend left]           node {luggage} (1/0)
%       (0) edge  [bend right=100]            node {no luggage} (3)
%       (0) edge  [loop left]            node {aisle occupied} (0)
%       (1/0) edge  [loop left]            node {can't see seat} (1/0)
%       (1/0) edge    [bend left]          node {no space by seat} (1/1)
%       (1/0) edge   [bend left]           node {space by seat} (2/0)
%       (1/1) edge  [loop left]            node {no space} (1/1)
%       (1/1) edge   [bend left]           node {arrived at back} (1/2)
%       (1/1) edge   [bend left]            node {free space} (2/1)
%       (1/2) edge   [loop left]           node {no space} (1/2)
%       (1/2) edge    [bend left]           node {free space} (2/2)
%       (2/0) edge     [out=330,in=300,looseness=8]        node {still storing} (2/0)
%       (2/0) edge    [bend left]          node {luggage $>$ capacity} (1/0)
%       (2/0) edge     [bend right]         node {stored all} (3)
%       (2/1) edge      [out=85,in=40,looseness=5]        node {still storing} (2/1)
%       (2/1) edge     [bend left]         node {luggage $>$ capacity} (1/1)
%       (2/1) edge              node {stored all} (3)
%       (2/2) edge     [out=85,in=40,looseness=5]         node {still storing} (2/2)
%       (2/2) edge     [bend left]         node {luggage $>$ capacity} (1/2)
%       (2/2) edge     [bend left]         node {stored all} (3)
%       (3) edge     [out=85,in=40,looseness=5]       node {not at seat} (3)
%       (3) edge     [bend right]         node {at seat} (4)
%       (4) edge      [out=85,in=40,looseness=5]         node {still sitting down} (4)
%       (4) edge              node {sat down} (5);
%
%\end{tikzpicture}
\end{figure}
 



\section{Simulation Results and Discussion}

\section{Summary and Outlook}

\section{References}






\end{document}  



 
